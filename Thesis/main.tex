\documentclass[withoutpreface,bwprint]{cumcmthesis}
\usepackage{bm}
\usepackage{tablefootnote}
\usepackage{longtable}
\usepackage[linesnumbered,ruled,vlined]{algorithm2e}

\title{\zihao{3}{\heiti{2023年第二届“钉钉杯”大学生 \\ 大数据挑战赛论文}}}
\tihao{A}
\baominghao{2023070519067}
\schoolname{武汉科技大学}
\membera{常璐瑶}
\memberb{杜睿}
\memberc{韦钧舰}
\supervisor{}
\yearinput{2023}
\monthinput{07}
\dayinput{15}

\begin{document}
\pagestyle{empty}
\maketitle

\medskip
\begin{tabularx}{0.9\textwidth}{lc}
    \zihao{4}{题 \qquad 目:}\hspace{1em} & \zihao{4}{\hspace{3em} 智能手机用户监测数据分析 \hspace{3em}} \\[-2pt]
    \cline{2-2}
\end{tabularx}

\newblock

\begin{abstract}

    经过数据探索,发现原始数据存在一定的错误、冗余及缺失。数据清洗后,尝试从原始数据(数值、类别、时间)中提炼统计特征,以便于问题一和问题二的解决。

    本节首先选取合适量化指标预处理,然后分别采用原型(K-Means++)、密度(DBSCAN)、层次(AGNES)三种聚类算法对用户进行聚类,遵循“肘部法则”选择合理的聚类数量K值;最后,根据聚类结果对不同类别的用户画像,分析不同群体用户的特征。

    cumcmthesis 是为全国大学生数学建模竞赛编写的\LaTeX{}模板, 旨在让大家专注于 论文的内容写作, 而不用花费过多精力在格式的定制和调整上.
    本手册是相应的参考, 其 中提供了一些环境和命令可以让模板的使用更为方便. 同时需要注意, 使用者需要有一 定的 \LaTeX{} 的使用经验,
    至少要会使用常用宏包的一些功能, 比如参考文献,数学公式,图片使用,列表环境等等. 例子文件参看 \texttt{example.tex}.

    \begin{mdframed} [%
            roundcorner=5pt,
            linecolor=gray!50,
            outerlinewidth=0.5pt,
            middlelinewidth=0.3pt, backgroundcolor=gray!2,
            innertopmargin=\topskip, frametitle={2020 年建模比赛格式变化说明},
            frametitlefont= \bfseries,frametitlerule=true,frametitlealignment =\raggedright\noindent,
            frametitlerulewidth=.5pt, frametitlebackgroundcolor=gray!2,]
        今年的格式变化主要就是三个地方,如下:
        \begin{enumerate}
            \item 论文第一页为承诺书,\textbf{\color{red}内容进行了调整}。

            \item 编号页格式进行了格式调整。

            \item 这是19年调整了,这里延续说明下。论文正文(\textbf{\color{red}不要目录},尽量控制在20页以内);正文之后是论文附录(页数不限)。

        \end{enumerate}

    \end{mdframed}

    \url{https://www.latexstudio.net} 陆续推出了更优质的资源,欢迎学习 。

    欢迎大家到QQ群里沟通交流:91940767/478023327/640633524。我们也开通了问答区交流 \LaTeX{}技术:\url{https://ask.latexstudio.net},欢迎大家前来交流,有问题就来这里,与大神零距离。

    \uwave{关注我们的微信公众号}:

    \centerline{\includegraphics[width=6cm]{gongzhonghao2}}

    \keywords{\TeX{}\quad  图片\quad   表格\quad  公式}
\end{abstract}

%目录
\tableofcontents
\newpage

\pagestyle{mainmatterstyle}
\setcounter{page}{1}
\section{问题重述}

\subsection{问题背景}

智能手机已成为现代社会人们生活不可或缺的一部分,其普及和发展给人们带来了巨大的生活便利和娱乐享受。近年中国智能手机市场品牌竞争进一步加剧,中国超越美国成为全球第一大智能手机市场。随着智能手机市场快速增长,智能手机用户群体愈发多样,智能手机软件满目琳琅,研究智能手机用户的行为模式和使用偏好对于理解用户需求、预测用户行为和优化产品与服务具有重要意义。通过对智能手机用户监测数据的分析,可以为智能手机制造商、软件开发者、广告商和营销人员等提供有益的信息及有价值的洞察,指导他们制定战略和决策,更好地贴合用户需求并提供更佳的用户体验。

\subsection{问题要求}

\begin{description}
    \item[问题一] 针对问题一,赛题要求(1)根据用户常用所属的20类APP的数据对用户聚类,(2)对不同类别的用户画像,分析不同群体用户的特征。
    \item[]
\end{description}

\newpage
\section{数据探索、预处理与特征提取}

\subsection{符号说明与数据概览}

原始数据集包含app类别辅助表格($A$.$\text{app\_class.csv}$)与21天监测数据($B^*$.$\text{dayxx.txt}$),来源、符号、意义及数据类型如下\cref{tab:001}~所示。

\begin{table}[!htbp]
    \caption{数据集原始特征}\label{tab:001} \centering
    \begin{tabular}{ccccc}
        \toprule[1.5pt]
        来源       & 符号            & 意义              & 类型   \\
        \midrule[1pt]
        $A\&B^*$ & $appid$       & 用户的id,唯一标识一名用户  & 类别变量 \\
        $A$      & $app\_class$  & 应用的id,唯一标识一个APP & 类别变量 \\
        $B^*$    & $app\_type$   & APP类型:系统自带、用户安装 & 类别变量 \\
        $B^*$    & $start\_day$  & 使用起始天,取值1-30    & 数值变量 \\
        $B^*$    & $start\_time$ & 使用起始时间          & 时间变量 \\
        $B^*$    & $end\_day$    & 使用结束天           & 数值变量 \\
        $B^*$    & $end\_time$   & 使用结束时间          & 时间变量 \\
        $B^*$    & $duration$    & 使用时长(秒)         & 数值变量 \\
        $B^*$    & $up\_flow$    & 上行流量            & 数值变量 \\
        $B^*$    & $down\_flow$  & 下行流量            & 数值变量 \\
        \bottomrule[1.5pt]
    \end{tabular}
\end{table}

将$B^*$与$A$进行“左连接”得到$B$(同时舍弃重复值),$app\_class$为a~t的代表$A$中20个常用类别($B$含3931种);$NaN$则代表所属类别未知的不常用APP($B$含32506种)。$B$中,在类别已知的常用20类APP中,t类数量最多(1406),r类最少(41)。

\begin{figure}[!htbp]
    \centering
    \includegraphics[width=.83\textwidth]{app_class_countplot_in_days}
    \caption{B中各类APP计数图}
    \label{fig:001}
\end{figure}

\subsection{数据探索之类别变量}

\subsubsection{单一变量}
\begin{table}[!htbp]
    \caption{类别变量统计描述(以day01为例)}\label{tab:002} \centering
    \begin{tabular}{llrll}
        \toprule[1.5pt]
               & $uid$                            & $appid$ & $app\_type$ & $app\_class$ \\
        \midrule[1pt]
        count  & 5335803                          & 5335803 & 5335803     & 5335803      \\
        unique & 35451                            & 11021   & 4           & 21           \\
        top    & A9E4AAC5B8E05D2A4E35E0D4F2994F37 & 3309    & usr         & NaN          \\
        freq   & 2629                             & 924309  & 2987468     & 2432606      \\
        \bottomrule[1.5pt]
    \end{tabular}
\end{table}

据\cref{tab:002}~,$app\_type$只有两类【系统预装、用户安装】,存在异常。通过数据探索,发现表格存在['sys', 'usr', '用户', '预装']四种取值,故将中文全部替换成英文。

$app\_class$有$21$类,这是因为在“左连接”操作时,将$NaN$也作为一种APP类型,这是由于此处数据缺失本身就代表一种资讯(小众APP),并非随机发生或人为故意。如果将$NaN$ 也视作一种$app\_class$,那么数据$B$ 不存在缺失值。

\subsubsection{天数变化}
另外,从21天的类型变量数据可以发现,每日活跃用户、APP、日志条数在各天都有所差异,如\cref{fig:002}~所示。

\begin{figure}[!htbp]
    \centering
    \begin{minipage}[c]{0.3\textwidth}
        \centering
        \includegraphics[width=0.95\textwidth]{relplot_line_day_val_logs}
        \subcaption{日志条数}
        \label{fig:002-a}
    \end{minipage}
    \begin{minipage}[c]{0.3\textwidth}
        \centering
        \includegraphics[width=0.95\textwidth]{relplot_line_day_val_users}
        \subcaption{活跃用户数}
        \label{fig:002-b}
    \end{minipage}
    \begin{minipage}[c]{0.3\textwidth}
        \centering
        \includegraphics[width=0.95\textwidth]{relplot_line_day_val_apps}
        \subcaption{活跃APP数}
        \label{fig:002-c}
    \end{minipage}
    \caption{day01~day21类别变量的变化折线图}
    \label{fig:002}
\end{figure}

粗略观察\cref{fig:002-b},活跃用户数在$7$和$21$存在明显波谷,似乎和“星期”有某种关联;对照\cref{fig:002-a}、\cref{fig:002-b}、\cref{fig:002-c}三表分析,似乎APP活跃情况(种类、请求)与天数有所关联,甚至可以猜测某些小众APP被某些特定用户群体所使用甚至是青睐。

\subsubsection{变量关联}

APP自身包含$appid$、$app\_class$以及$app\_type$属性,因此可以抽取这三列建立$C$。

\begin{table}[!htbp]
    \caption{APP统计描述}\label{tab:003} \centering
    \begin{tabular}{lrll}
        \toprule
               & $appid$ & $app\_type$ & $app\_class$ \\
        \midrule
        count  & 37276   & 37276       & 37276        \\
        unique & 36437   & 2           & 21           \\
        top    & 19582   & usr         & NaN          \\
        freq   & 2       & 34153       & 32958        \\
        \bottomrule
    \end{tabular}
\end{table}

据\cref{tab:003}~,数据探索发现有839个app多重类型,即对于某些用户而言,该软件为系统预装,对另一些而言,则为自行安装。这似乎表明,不同的用户在软件下载与安装层面,有可相互区分的行为特征。另外,21类APP的数量情况如\cref{fig:003}~所示。

\begin{figure}[!htbp]
    \centering
    \includegraphics[width=.95\textwidth]{catplot_count_app_class}
    \caption{21类APP计数图}
    \label{fig:003}
\end{figure}

据\cref{fig:003}~,未分类的APP并不是小数目(有一部分为系统预装),$t$类在APP多样性表现上依然出众。此外,不同种类的APP就类型(预装、用户)而言相差较为悬殊,例如:$g$类,系统预装相对较多;其余类别,用户预装较为普遍,尤其是$d$和$r$类。

值得区分的是,在新建APP统计数据$C$中,计数图反映了各类APP的多样程度(市场垄断程度)。而在监测数据$B$中,某个(类)APP请求日志(行数)计数图则反映了该用户在使用各类APP时的活跃(点击)行为,这将在下一小节进行探索。

\subsection{数据探索之数值变量}

\subsubsection{单一变量}

\begin{table}[!htbp]
    \caption{数值变量统计描述(以day01为例)}\label{tab:004} \centering
    \begin{tabular}{lrrrrr}
        \toprule[1.5pt]
              & $start\_day$ & $end\_day$ & $duration$     & $up\_flow$      & $down\_flow$   \\
        \midrule[1pt]
        count & 5335803      & 5335803    & 5335803        & 5335803         & 5335803        \\
        mean  & 0.975107     & 1          & 2151.604772    & 607572.168995   & 158163.759549  \\
        std   & 16.843899    & 0          & 1455335.155631 & 11015502.975274 & 6538529.614936 \\
        min   & -16524       & 1          & 1              & 0               & 0              \\
        25\%  & 1            & 1          & 3              & 0               & 0              \\
        50\%  & 1            & 1          & 10             & 0               & 0              \\
        75\%  & 1            & 1          & 36             & 1278            & 1063           \\
        max   & 1            & 1          & 1427769883     & 3639473769      & 3292713011     \\
        \bottomrule[1.5pt]
    \end{tabular}
\end{table}

数据具体说明指出:“$start\_day$:使用起始天,取值1-30(注:第一天数据的头两行的使用起始天取值为0,说明是在这一天的前一天开始使用的)。”然而,\cref{tab:004}~显示其最小值为$-16542$,因此可以判断$start\_day$存在异常值;而这会直接导致$duration$、$up\_flow$、$down\_flow$偏差过大。因此,须要对这两列进行数据清洗,删除异常值。

\begin{figure}[!htbp]
    \centering
    \includegraphics[width=.95\textwidth]{boxenplot_duration_day01}
    \caption{使用时长、上下行流量增强箱形图(以day01为例)}
    \label{fig:004}
\end{figure}

据统计,在第1~21天监测数据中,$99.98\%$的记录使用时长不超过$9158.02$。针对异常案例进行分析,例如,$uid=64B3E40461C56847F35DB46D55707EA4$用户:

\begin{table}[!htbp]
    \caption{异常案例}\label{tab:005} \centering
    \begin{tabular}{lllllrlrlrrr}
        \toprule[1.5pt]
        appid & app\_class & start\_day & start\_time & end\_day & end\_time & duration \\
        \midrule[1pt]
        4803  & a          & 19         & 00:52:46    & 19       & 07:47:59  & 24912    \\
        18478 & c          & 19         & 07:48:25    & 19       & 07:48:38  & 12       \\
        \.    & \.         & \.         & \.          & \.       & \.        & \.       \\
        6192  & NaN        & 19         & 20:46:11    & 19       & 20:46:29  & 18       \\
        3309  & f          & 19         & 23:14:49    & 19       & 23:16:38  & 109      \\
        \bottomrule[1.5pt]
    \end{tabular}
\end{table}

凌晨零时至早上七时的记录是不符合生活常态不可持续的,猜测是应用后台驻留、系统故障或用户因故未关闭应用。不过,是否存在异常行为可以作为一个新的特征。因此,本文对持续时间超过$9159$的认定为无效使用时长,不能真实反映用户的行为特征。

此外,以第一天为例,不同用户在$appid$个数、$duration$总时长、$up\_flow$、$down\_flow$总流量、日志行数层面有所差异,分布如\cref{fig:005}~所示。

\begin{figure}[!htbp]
    \centering
    \includegraphics[width=.95\textwidth]{histplots_appid_duration_upflow_downflow_lines}
    \caption{直方图(使用APP数量、使用有效总时长、消耗上下行流量、日志行数)}
    \label{fig:005}
\end{figure}

可以观察到某些用户较依赖手机,日志数、使用时长、APP多样性、消耗流量偏多。

\subsubsection{天数变化}

探索发现,可为每名用户绘制使用APP总数/时长/流量/日志随天数变化的折线图,以展现用户在月(周)级别的变化趋势与独潜在规律,并据此将用户群归类。

\subsubsection{变量关联}

更细粒度地,对于每名用户/每天,可绘制其各类APP的个数/时长/流量/日志情况,以了解用户对不同类型APP的适用情况、青睐程度;更进一步,还可以绘制24小时各类APP使用情况,这样便于了解用户的作息、通勤、活跃时段等信息。

\subsection{直觉小结与用户量化}

\subsubsection{数据直觉}
概括来说,赛题数据$A$给出4000多个常用APP所属类别:这是$1:1$的数据;而赛题数据$B$则记录了每名用户($uid$)每日每时使用各款APP($appid$)的起始时间,使用时长,上下流量等信息:这是$1:N$的资料。可以使用统计量进行归纳。

具体而言,监测数据蕴藏大量用户行为特征,例如:
\begin{itemize}
    \item 用户一天使用多长时间的手机(可间接反映依赖程度、年龄)
    \item 用户平均多长时间看一次手机(可间接反映依赖程度、年龄、工作)
    \item 距离上一次上线,隔了几天(可间接反映依赖程度、年龄、工作)
    \item 在什么星期几的时间段最常用什么类型APP(可反映工作、生活)
    \item 用户早、晚各使用什么类APP(可反映工作、生活、作息)
    \item 用户周末、工作日各使用什么类APP(可反映工作、生活、作息)
    \item 用户最早什么时候开始使用手机、什么时候结束使用(可反映工作、作息)
    \item 用户最常用什么类型APP(可反映喜好)
    \item 哪一类APP使用最频繁、哪一类使用时长最多(可反映喜好)
    \item 用户一共安装了多少个APP(可反映保守程度、对多样性的接纳程度)
    \item 系统预装与自行安装的比例(可反映保守程度、对多样性的接纳程度)
    \item 用户的每月流量使用情况(可间接反映财富程度、年龄)
\end{itemize}

\subsubsection{用户模型}

用户量化是指将现实生活中的“用户实体”进行抽象,采用不同维度的量化指标建模,即将其视为$n$维空间的一个点,使用形如$X=[x_1, x_2, \cdots , x_n]$的数学符号表示。

基于对数据集的深入探索及理解,提出简易用户模型:

\begin{table}[!htbp]
    \caption{简易用户模型}\label{tab:002} \centering
    \begin{tabular}{lllllrlrlrrr}
        \toprule[1.5pt]
        符号             & 意义                               & 维度                   \\
        \midrule[1pt]
        $uid$          & 用户的唯一标识                          & $1$                  \\
        $DU_{d, h, c}$ & 该用户在第$d$天$h$时内使用$c$类APP时,投入的总计时长 & $d\times h \times c$ \\
        $UF_{d, h, c}$ & 该用户在第$d$天$h$时内使用$c$类APP时,消耗的上行流量 & $d\times h \times c$ \\
        $DF_{d, h, c}$ & 该用户在第$d$天$h$时内使用$c$类APP时,消耗的下行流量 & $d\times h \times c$ \\
        $NO_{d, h, c}$ & 该用户在第$d$天$h$时内使用$c$类APP时,记录的日志行数 & $d\times h \times c$ \\
        \bottomrule[1.5pt]
    \end{tabular}
\end{table}

\section{问题一:聚类分析与用户画像}

\subsection{特征工程与评价指标}

\subsubsection{特征选择与数据降维}

聚类指将数据样本对象划分成若干类(簇、标签)并尽可能的保证“类内紧凑”、“类间独立”\cite{数据仓库与数据挖掘}。不同的量化指标、不同的相似度量(距离定义),往往会带来迥异的聚类结果。一般来说,量化指标维度数目越多,算法运行时间越长、结论可解释性越弱。

关于用户画像的量化指标,陈\cite{陈纯}等人、成\cite{成雪}等人从各类日均屏幕使用时间切入;武\cite{武慧娟}等人从APP数量、阅读时间、消费等特征对阅读类APP使用人群进行聚类解读;侯\cite{侯金凤}针对每日手机使用时长、使用频次、使用偏好等特征对用户进行建模;韦\cite{韦磊}基于“安装数量”、“打开次数”、“使用时长”、“工作日使用时长”、“周末使用时长”构建用户特征。

首先,尝试选择前7日各类APP使用时长、使用频次、上行流量、下行流量之和作为量化特征,共计$20*4=80$维;对于不同的量纲特征,分别扣除均值,除以标准差以进行数值标准化。Pearson相关热力\cref{fig:pearson}~显示,各类APP时长、频次、流量成弱正相关。

\begin{figure}[!htbp]
    \centering
    \includegraphics[width=.91\textwidth]{heatmap_features}
    \caption{皮尔逊相关系数热力图}
    \label{fig:pearson}
\end{figure}

为增加数据易用性,降低计算开销,增强视觉理解,而后采用主成分分析对特征进行变换,并按方差排序表示各维度重要程度,如\cref{fig:bar-pca}~,选定阈值将维度压缩至$3$维。

\begin{figure}[!htbp]
    \centering
    \includegraphics[width=.91\textwidth]{barplot_pca_features}
    \caption{PCA特征方差柱状图}
    \label{fig:bar-pca}
\end{figure}

\subsubsection{评价指标}

聚类“好坏”不存在绝对的客观的标准\cite{机器学习};聚类数目设定是否“合理”也往往依赖人工先验知识\cite{高维数据的聚类分析}。聚类数目设定过低,划分粒度不够细腻;聚类数目设定过高,宏观结论的可解释性又受到限制。常用选择聚类数目方法是人为观察聚合系数折线图,大致估计最优聚类数量$K$。相关定义如下:

\begin{definition}
    \textbf{各簇畸变程度}:该簇重心与其内部成员位置距离的平方和;
    假设一共将$n$个样本划分到$K$个簇中,用$C_k$表示第$k$簇,该簇重心记为$u_k$,则第$k$簇的畸变程度为:$$\sum_{i\in C_k} |x_i-u_k|^2$$
    \label{def:001}
\end{definition}

\begin{definition}
    \textbf{聚合系数}:$$J=\sum_{k=1}^{K} \sum_{i \in C_k} |x_i-u_k|^2$$
    \label{def:002}
\end{definition}

此外,还有Calinski-Harabasz系数\cite{CH}、Davies-Bouldin指数\cite{DB}、Silhouette轮廓系数\cite{SC}可用于度量某些聚类目的下的结论性能。

\subsection{算法概述与K值选择}

注:本小节所使用算法及评价指标均采用scikit\_learn\cite{sklearn}开源库实现。

\subsubsection{原型聚类:K-Means++}

K-Means是一种简单、高效的聚类算法,假设聚类结构能通过一组“原型”刻画,算法的主要思想是通过迭代过程把数据集划分为不同的类别,流程如\cref{fig:008}~。K-means++优化“初始化K个聚类中心”,要求初始的聚类中心之间的相互距离要尽可能的远,在“孤立点数据敏感性”方面优于K-Means算法。默认采用欧式距离、重心法进行相似度量。

\begin{figure}[!htbp]
    \centering
    \includegraphics[width=.37\textwidth]{process_kmeans}
    \caption{KMeans算法流程图}
    \label{fig:008}
\end{figure}

将最大迭代次数设置为1000,选择K等于2~50绘制聚合系数与卡林斯基-哈拉巴斯指数折线图。根据\cref{fig:009}~,K值从2到13时,畸变程度变化最大;超过6畸变程度变化显著降低:因此根据肘部法则,可将聚类数量$K$设定为5;从来看,应将聚类数量设定为6以下。该结论符合卡林斯基-哈拉巴斯指数峰值,故将聚合数目设定为$5$。

\begin{figure}[!htbp]
    \centering
    \includegraphics[width=.77\textwidth]{lineplot_Ks_inertias_CH.pdf}
    \caption{K-Means 聚合系数与卡林斯基-哈拉巴斯指数}
    \label{fig:009}
\end{figure}

\subsubsection{层次聚类:AGNES}

AGNES算法(Agglomerative
Nesting),以自底向上方式,不断重复合并,产生不同粒度(层次)的聚类结果,一般最终预设聚类数目为1。该算法可通过聚类谱系图(dendrogram)可视化,算法执行流程如下:

\IncMargin{1em}
\begin{algorithm} \SetKwData{Left}{left}\SetKwData{This}{this}\SetKwData{Up}{up} \SetKwFunction{Union}{Union}\SetKwFunction{FindCompress}{FindCompress} \SetKwInOut{Input}{input}\SetKwInOut{Output}{output}

    \Input{样本集$D=\{ x_1, x_2, \ldots, x_m \}$; \\
        聚类簇距离度量函数$d$;\\
        聚类簇数$k$。}
    \Output{簇划分:$\mathcal{C}=\{ C_1, C_2, \ldots, C_k \}$ }
    \BlankLine

    \BlankLine
    \emph{\#先将每个样本视作一个初始簇构造}\;
    \emph{\#构造$M$个类,每个类仅包含一个样本}\;
    \For{$j = 1,2, \ldots, m$}{
    $C_j={x_j}$
    }

    \BlankLine
    \emph{\#两两计算距离}\;
    \For{$i = 1,2, \ldots, m$}{
    \For{$j = i+1, \ldots, m$}{
    $M_{i, j}=d(C_i, C_j)$\;
    $M_{j, i}=M_{i, j}$
    }
    }

    \BlankLine
    \emph{\#当前类个数大于预设簇数}\;
    \While{$q > k$}{

    合并距离最近的两个聚类簇$C_{i^*}= C_{i^*} \cup C_{j^*}$\;
    \For{$j=j^*+1, j=j^*+2, \ldots, q$}{
        将聚类簇$C_{j}$重编号为$C_{j-1}$
    }

    删除距离矩阵$M$的第$j*$行与第$j*$列\;
    \# 重新计算距离矩阵\;
    \For{$j=1,2, \ldots, q-1$}{
    $M_{i^*, j}=d(C_{i^*}, C_{j})$\;
    $M_{j, i^*}=M_{i^*, j}$
    }
    q=q-1
    }
    \caption{AGNES算法}
    \label{algo:001}
\end{algorithm}
\DecMargin{1em}

默认采用“欧式距离”进行度量样本距离,采用“离差平方和”(ward linkage)作为簇距离度量函数。该算法执行结果见下页:

\subsubsection{密度聚类:DBSCAN}

DBSCAN算法从样本密度的角度来考察样本之间的可连接性,要求聚类空间中的以$eps$为半径的邻域内所包含对象的数目不小于某一给定阈值$min\_samples$,并基于可连接样本不断扩展生长聚类簇以获得最终的聚类结果。DBSCAN不需要预先输入要划分的聚类个数,但是对$eps$、$min\_samples$参数敏感。记特征维度数目$N=3$、$K=2N-1$,按照以下经验值确定超参数\cite{eps1,eps2}:$min\_samples=2N=6$,将数据集各点与K-最近邻算法分类标签的距离排序,观察\cref{fig:008}~拐点$y$坐标确定$eps=2$。

\begin{figure}[!htbp]
    \centering
    \includegraphics[width=.37\textwidth]{relplot_knn_distances}
    \caption{数据集各点6-最近邻距离(排序)}
    \label{fig:010}
\end{figure}

运行结果:聚类数量为$7$,噪点用户$539$名。

\subsection{算法比较与用户画像}

共性。

\newpage
\section{问题二:未来使用情况预测}

\subsection{问题分析与流程思路}

XGBoost、ResNet、

调整参数

ARIMA

固定窗

滑动窗

验证

\newpage
\subsection{你好}
要使用 \LaTeX{} 来完成建模论文,首先要确保正确安装一个 \LaTeX{} 的发行版本。

\begin{itemize}
    \item Mac 下可以使用 Mac\TeX{}
    \item Linux 下可以使用 \TeX{}Live ;
    \item windows 下可以使用 \TeX{}Live 或者 Mik\TeX{} ;
\end{itemize}

具体安装可以参考 \href{https://github.com/OsbertWang/install-latex-guide-zh-cn/releases/}{Install-LaTeX-Guide-zh-cn} 或者其它靠谱的文章。另外可以安装一个易用的编辑器,例如 \href{https://mirrors.tuna.tsinghua.edu.cn/github-release/texstudio-org/texstudio/LatestRelease/}{\TeX{}studio} 。

使用该模板前,请阅读模板的使用说明文档。下面给出模板使用的大概样式。

\begin{tcode}
    \documentclass{cumcmthesis}
    %\documentclass[withoutpreface,bwprint]{cumcmthesis} %去掉封面与编号页

    \title{论文题目}
    \tihao{A}            % 题号
    \baominghao{4321}    % 报名号
    \schoolname{你的大学}
    \membera{成员A}
    \memberb{成员B}
    \memberc{成员C}
    \supervisor{指导老师}
    \yearinput{2017}     % 年
    \monthinput{08}      % 月
    \dayinput{22}        % 日

    \begin{document}
    \maketitle
    \begin{abstract}
        摘要的具体内容。
        \keywords{关键词1\quad  关键词2\quad   关键词3}
    \end{abstract}
    \tableofcontents
    \section{问题重述}
    \subsection{问题的提出}
    \section{模型的假设}
    \section{符号说明}
    \begin{center}
        \begin{tabular}{cc}
            \hline
            \makebox[0.3\textwidth][c]{符号} & \makebox[0.4\textwidth][c]{意义} \\ \hline
            D                              & 木条宽度(cm)                       \\ \hline
        \end{tabular}
    \end{center}
    \section{问题分析}
    \section{总结}
    \begin{thebibliography}{9}%宽度9
        \bibitem{bib:one} ....
    \end{thebibliography}
    \begin{appendices}
        附录的内容。
    \end{appendices}
    \end{document}
\end{tcode}

根据要求,电子版论文提交时需去掉封面和编号页。可以加上 \verb|withoutpreface|  选项来实现,即:
\begin{tcode}
    \documentclass[withoutpreface]{cumcmthesis}
\end{tcode}
这样就能实现了。打印的时候有超链接的地方不需要彩色,可以加上 \verb|bwprint| 选项。

另外目录也是不需要的,将 \verb|\tableofcontents| 注释或删除,目录就不会出现了。

团队的信息填入指定的位置,并且确保信息的正确性,以免因此白忙一场。

编译记得使用 \verb|xelatex|,而不是用 \verb|pdflatex|。在命令行编译的可以按如下方式编译:
\begin{tcode}
    xelatex example
\end{tcode}
或者使用 \verb|latexmk| 来编译,更推荐这种方式。
\begin{tcode}
    latexmk -xelatex example
\end{tcode}

下面给出写作与排版上的一些建议。

\section{图片}

建模中不可避免要插入图片。图片可以分为矢量图与位图。位图推荐使用 \verb|jpg,png| 这两种格式,避免使用 \verb|bmp| 这类图片,容易出现图片插入失败这样情况的发生。矢量图一般有 \verb|pdf,eps| ,推荐使用 \verb|pdf|  格式的图片,尽量不要使用 \verb|eps| 图片,理由相同。

注意图片的命名,避免使用中文来命名图片,可以用英文与数字的组合来命名图片。避免使用\verb|1,2,3| 这样顺序的图片命名方式。图片多了,自己都不清楚那张图是什么了,命名尽量让它有意义。下面是一个插图的示例代码。
\begin{figure}[!h]
    \centering
    \includegraphics[width=.6\textwidth]{smokeblk}
    \caption{电路图}
    \label{fig:circuit-diagram}
\end{figure}

注意 \verb|figure| 环境是一个浮动体环境,图片的最终位置可能会跑动。\verb|[!h]| 中的 \verb|h| 是 here 的意思, \verb|!| 表示忽略一些浮动体的严格规则。另外里面还可以加上 \verb|btp| 选项,它们分别是 bottom, top, page 的意思。只要这几个参数在花括号里面,作用是不分先后顺序的。page 在这里表示浮动页。

\verb|\label{fig:circuit-diagram}| 是一个标签,供交叉引用使用的。例如引用图片 \verb|\cref{fig:circuit-diagram}| 的实际效果是\cref{fig:circuit-diagram}。图片是自动编号的,比起手动编号,它更加高效。\verb|\cref{label}| 由 \verb|cleveref| 宏包提供,比普通的 \verb|\ref{label}| 更加自动化。 \verb|label| 要确保唯一,命名方式推荐用图片的命名方式。

图片并排的需求解决方式多种多样,下面用 \verb|minipage| 环境来展示一个简单的例子。注意,以下例子用到了 \verb|subcaption| 命令,需要加载 subcaption 宏包。

这相当于整体是一张大图片,大图片引用是\cref{fig:sample-figure},子图引用别分是\cref{fig:sample-figure-a}、\cref{fig:sample-figure-b}、\cref{fig:sample-figure-c}。

如果原本两张图片的高度不同,但是希望它们缩放后等高的排在同一行,参考这个例子:
\begin{figure}
    \centering
    \begin{minipage}[c]{0.48\textwidth}
        \centering
        \includegraphics[height=0.2\textheight]{cat}
        \subcaption{一只猫}
    \end{minipage}
    \begin{minipage}[c]{0.48\textwidth}
        \centering
        \includegraphics[height=0.2\textheight]{smokeblk}
        \subcaption{电路图}
    \end{minipage}
    \caption{多图并排示例}
\end{figure}

\section{绘制普通三线表格}
表格应具有三线表格式,因此常用 booktabs宏包,其标准格式如\cref{tab:404}~所示。
\begin{table}[!htbp]
    \caption{标准三线表格}\label{tab:404} \centering
    \begin{tabular}{ccccc}
        \toprule[1.5pt]
        $D$(in) & $P_u$(lbs) & $u_u$(in) & $\beta$ & $G_f$(psi.in) \\
        \midrule[1pt]
        5       & 269.8      & 0.000674  & 1.79    & 0.04089       \\
        10      & 421.0      & 0.001035  & 3.59    & 0.04089       \\
        20      & 640.2      & 0.001565  & 7.18    & 0.04089       \\
        \bottomrule[1.5pt]
    \end{tabular}
\end{table}

其绘制表格的代码及其说明如下。
\begin{tcode}
    \begin{table}[!htbp]
        \caption[标签名]{中文标题}
        \begin{tabular}{cc...c}
            \toprule[1.5pt]
            表头第1个格    & 表头第2个格    & ... & 表头第n个格                \\
            \midrule[1pt]
            表中数据(1,1) & 表中数据(1,2) & ... & 表中数据(1,n)             \\
            表中数据(2,1) & 表中数据(2,2) & ... & 表中数据(2,n)             \\
            ................................................... \\
            表中数据(m,1) & 表中数据(m,2) & ... & 表中数据(m,n)             \\
            \bottomrule[1.5pt]
        \end{tabular}
    \end{table}
\end{tcode}

\bigskip

table环境是一个将表格嵌入文本的浮动环境。tabular环境的必选参数由每列对应一个格式字符所组成:c表示居中,l表示左对齐,r表示右对齐,其总个数应与表的列数相同。此外,\verb|@{文本}|可以出现在任意两个上述的列格式之间,其中的文本将被插入每一行的同一位置。表格的各行以\verb|\\|分隔,同一行的各列则以\&分隔。
\verb|\toprule| 、\verb|\midrule| 和 \verb|\bottomrule| 三个命令是由booktabs宏包提供的,其中
\verb|\toprule| 和 \verb|\bottomrule| 分别用来绘制表格的第一条(表格最顶部)和第三条(表格最底部)水平线,
\verb|\midrule| 用来绘制第二条(表头之下)水平线,且第一条和第三条水平线的线宽为 1.5pt ,第二条水平线的线宽为 1pt
。引用方法与图片的相同。

\section{公式}

数学建模必然涉及不少数学公式的使用。下面简单介绍一个可能用得上的数学环境。

首先是行内公式,例如 $ \theta $ 是角度。行内公式使用 \verb|$  $| 包裹。

行间公式不需要编号的可以使用 \verb|\[  \]| 包裹,例如
\[
    E=mc^2
\]
其中 $ E $ 是能量,$ m $ 是质量,$ c $ 是光速。

如果希望某个公式带编号,并且在后文中引用可以参考下面的写法:
\begin{equation}
    E=mc^2
    \label{eq:energy}
\end{equation}
式\cref{eq:energy}是质能方程。

多行公式有时候希望能够在特定的位置对齐,以下是其中一种处理方法。
\begin{align}
    P & = UI   \\
      & = I^2R
\end{align}
\verb|&| 是对齐的位置, \verb|&| 可以有多个,但是每行的个数要相同。

矩阵的输入也不难。
\[
    \mathbf{X} = \left(
    \begin{array}{cccc}
            x_{11} & x_{12} & \ldots & x_{1n} \\
            x_{21} & x_{22} & \ldots & x_{2n} \\
            \vdots & \vdots & \ddots & \vdots \\
            x_{n1} & x_{n2} & \ldots & x_{nn} \\
        \end{array} \right)
\]

分段函数这些可以用 \verb|case| 环境,但是它要放在数学环境里面。
\[
    f(x) =
    \begin{cases}
        0 & x \text{为无理数} , \\
        1 & x \text{为有理数} .
    \end{cases}
\]
在数学环境里面,字体用的是数学字体,一般与正文字体不同。假如要公式里面有个别文字,则需要把这部分放在 \verb|text| 环境里面,即 \verb|\text{文本环境}| 。

公式中个别需要加粗的字母可以用 \verb|$\bm{math symbol}$| 。如 $ \alpha a\bm{\alpha a} $ 。

以上仅简单介绍了基础的使用,对于更复杂的需求,可以阅读相关的宏包手册,如 \href{http://texdoc.net/texmf-dist/doc/latex/amsmath/amsldoc.pdf}{amsmath}。

希腊字母这些如果不熟悉,可以去查找符号文件 \href{http://mirrors.ctan.org/info/symbols/comprehensive/symbols-a4.pdf}{symbols-a4.pdf} ,也可以去 \href{http://detexify.kirelabs.org/classify.html}{detexify} 网站手写识别。另外还有数学公式识别软件 \href{https://mathpix.com/}{mathpix} 。

下面简单介绍一下定理、证明等环境的使用。

除了 definition 环境,还可以使用 theorem 、lemma、corollary、assumption、conjecture、axiom、principle、problem、example、proof、solution 这些环境,根据论文的实际需求合理使用。

\begin{theorem}
    这是一个定理。
    \label{thm:example}
\end{theorem}
由\cref{thm:example}我们知道了定理环境的使用。

\begin{lemma}
    这是一个引理。
    \label{lem:example}
\end{lemma}
由\cref{lem:example}我们知道了引理环境的使用。

\begin{corollary}
    这是一个推论。
    \label{cor:example}
\end{corollary}
由\cref{cor:example}我们知道了推论环境的使用。

\begin{assumption}
    这是一个假设。
    \label{asu:example}
\end{assumption}
由\cref{asu:example}我们知道了假设环境的使用。

\begin{conjecture}
    这是一个猜想。
    \label{con:example}
\end{conjecture}
由\cref{con:example}我们知道了猜想环境的使用。

\begin{axiom}
    这是一个公理。
    \label{axi:example}
\end{axiom}
由\cref{axi:example}我们知道了公理环境的使用。

\begin{principle}
    这是一个定律。
    \label{pri:example}
\end{principle}
由\cref{pri:example}我们知道了定律环境的使用。

\begin{problem}
这是一个问题。
\label{pro:example}
\end{problem}
由\cref{pro:example}我们知道了问题环境的使用。

\begin{example}
    这是一个例子。
    \label{exa:example}
\end{example}
由\cref{exa:example}我们知道了例子环境的使用。

\begin{proof}
    这是一个证明。
    \label{prf:example}
\end{proof}
由\cref{prf:example}我们知道了证明环境的使用。

\begin{solution}
    这是一个解。
    \label{sol:example}
\end{solution}
由\cref{sol:example}我们知道了解环境的使用。

\section{参考文献与引用}

参考文献对于一篇正式的论文来说是必不可少的,在建模中重要的参考文献当然应该列出。\LaTeX{}在这方面的功能也是十分强大的,下面进介绍一个比较简单的参考文献制作方法。有兴趣的可以学习 \verb|bibtex| 或 \verb|biblatex| 的使用。

\LaTeX{}的入门书籍可以看《\LaTeX{}入门》\cite{liuhaiyang2013latex}。这是一个简单的引用,用 \verb|\cite{bibkey}| 来完成。要引用成功,当然要维护好 bibitem 了。下面是个简单的例子。

%参考文献
\newpage
\begin{thebibliography}{9}
    \bibitem{机器学习} 周志华. 机器学习[M]. 北京. 清华大学出版社. 2016. 197-219
    \bibitem{高维数据的聚类分析} 何宏. 高维数据的聚类分析[M]. 上海. 上海交通大学出版社. 2022. 1-16
    \bibitem{数据仓库与数据挖掘} 陈志泊,韩慧,王建新,孙俏,聂耿青. 数据仓库与数据挖掘[M]. 北京. 清华大学出版社. 2009
    \bibitem{常乐} 常乐. 基于用户行为分析的用户画像系统设计与实现[D]. 北京邮电大学. 2020
    \bibitem{陈纯} 陈纯,龙瀛,黄贵恺. 屏幕使用时间与步行活动关系的探索性研究[J]. 景观设计学(中英文). 2021. 9(04):68-81
    \bibitem{成雪} 成雪,于冬梅,赵丽云等. 2016—2017年中国各省中小学生电子屏幕使用现状[J]. 卫生研究. 2023,52(03):382-387
    \bibitem{武慧娟} 武慧娟,赵天慧,孙鸿飞等. 基于支付意愿的数字阅读用户画像聚类研究[J]. 情报科学. 2022. 40(05)
    \bibitem{侯金凤} 侯金凤. 移动互联网下手机用户使用行为特征的研究[J]. 电脑知识与技术. 2016,12(07)
    \bibitem{韦磊} 韦磊. 基于移动终端数据的用户画像模型研究[D]. 江苏科技大学. 2021
    \bibitem{timeseries} Garg, R., \& Barpanda, S. Machine Learning Algorithms for Time Series Analysis and Forecasting[Z/OL]. arXiv preprint arXiv:2211.14387. https://arxiv.org/abs/2211.14387
    \bibitem{CH} T. Caliński, J Harabasz. A dendrite method for cluster analysis[J]. Communications in Statistics. 1974. 3:1, 1-27
    \bibitem{DB} Davies D L , Bouldin D W. A Cluster Separation Measure[J]. IEEE Transactions on Pattern Analysis and Machine Intelligence. 1979. PAMI-1(2):224-227
    \bibitem{SC} Peter R J . Silhouettes: A graphical aid to the interpretation and validation of cluster analysis[J]. Journal of Computational \& Applied Mathematics. 1987. 20
    \bibitem{sklearn} Swami A , Jain R. Scikit-learn: Machine Learning in Python[J]. Journal of Machine Learning Research. 2013, 12(10):2825-2830
    \bibitem{eps1} Sander J , Ester M , Kriegel H P ,et al. Density-Based Clustering in Spatial Databases: The Algorithm GDBSCAN and Its Applications[J]. Data Mining \& Knowledge Discovery. 1998. 2(2):169-194
    \bibitem{eps2} Schubert E , Sander J , Ester M ,et al. DBSCAN revisited, revisited: Why and how you should (still) use DBSCAN[J]. ACM Transactions on Database Systems, 2017, 42(3):1-21
    % \bibitem[6] 论文集: \[序号\] 著者.篇名.主编.论文集名\[C\].出版地.出版者.出版年.起止页码
    % \bibitem[7] 科技报告:\[序号\]著者.题名\[R\]. 报告题名. 编号.出版地.出版者. 出版年.起止页码
    % \bibitem[8] 学位论文:\[序号\] 著者.题名\[D\].保存地点.授予年
    % \bibitem[9] 专利文献:\[序号\] 专利申请者.题名\[P\]. 国别.专利文献种类.专利号.出版日期
\end{thebibliography}

%附录
\newpage
\begin{appendices}

    \section{环境依赖与使用说明}
    \begin{table}[htbp]
        \centering
        \caption{依赖罗列}
        \begin{tabular}{ccccc}
            \toprule
            \multicolumn{5}{c}{模板中已经加载的宏包}                      \\
            \midrule
            amsbsy & amsfonts & {amsgen} & {amsmath} & {amsopn} \\
            \bottomrule
        \end{tabular}%
        \label{tab:addlabel}%
    \end{table}%

    以上宏包都已经加载过了,不要重复加载它们。

    \section{2.1 源代码}

    \section{2.2 源代码}

    \section{2.3 源代码}

    \begin{lstlisting}[language=matlab]
kk=2;[mdd,ndd]=size(dd);
while ~isempty(V)
[tmpd,j]=min(W(i,V));tmpj=V(j);
for k=2:ndd
[tmp1,jj]=min(dd(1,k)+W(dd(2,k),V));
tmp2=V(jj);tt(k-1,:)=[tmp1,tmp2,jj];
end
tmp=[tmpd,tmpj,j;tt];[tmp3,tmp4]=min(tmp(:,1));
if tmp3==tmpd, ss(1:2,kk)=[i;tmp(tmp4,2)];
else,tmp5=find(ss(:,tmp4)~=0);tmp6=length(tmp5);
if dd(2,tmp4)==ss(tmp6,tmp4)
ss(1:tmp6+1,kk)=[ss(tmp5,tmp4);tmp(tmp4,2)];
else, ss(1:3,kk)=[i;dd(2,tmp4);tmp(tmp4,2)];
end;end
dd=[dd,[tmp3;tmp(tmp4,2)]];V(tmp(tmp4,3))=[];
[mdd,ndd]=size(dd);kk=kk+1;
end; S=ss; D=dd(1,:);
 \end{lstlisting}

    \section{规划解决程序--lingo源代码}

    \begin{lstlisting}[language=c]
kk=2;
[mdd,ndd]=size(dd);
while ~isempty(V)
    [tmpd,j]=min(W(i,V));tmpj=V(j);
for k=2:ndd
    [tmp1,jj]=min(dd(1,k)+W(dd(2,k),V));
    tmp2=V(jj);tt(k-1,:)=[tmp1,tmp2,jj];
end
    tmp=[tmpd,tmpj,j;tt];[tmp3,tmp4]=min(tmp(:,1));
if tmp3==tmpd, ss(1:2,kk)=[i;tmp(tmp4,2)];
else,tmp5=find(ss(:,tmp4)~=0);tmp6=length(tmp5);
if dd(2,tmp4)==ss(tmp6,tmp4)
    ss(1:tmp6+1,kk)=[ss(tmp5,tmp4);tmp(tmp4,2)];
else, ss(1:3,kk)=[i;dd(2,tmp4);tmp(tmp4,2)];
end;
end
    dd=[dd,[tmp3;tmp(tmp4,2)]];V(tmp(tmp4,3))=[];
    [mdd,ndd]=size(dd);
    kk=kk+1;
end;
S=ss;
D=dd(1,:);
 \end{lstlisting}
\end{appendices}

\end{document}